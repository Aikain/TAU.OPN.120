
\chapter{Ohjeistus Gemini-kielimallille}
\label{ch:gemini-guide}

Tulosteen tulee olla JSON-muodossa. Vuorosanojen `type` on `TALK` ja tekemisten
`ACTION`.

Esimerkki:
[\\
    \{
        "type": "ACTION",
        "target": "Äiti",
        "content": "tekee ruokaa kotona"
    \},\\
    \{
        "type": "ACTION",
        "target": "Ellu",
        "content": "saapuu koulusta kotia"
    \},\\
    \{
        "type": "TALK",
        "target": "Ellu",
        "content": "Moi äiti!"
    \},\\
    \{
        "type": "TALK",
        "target": "Äiti",
        "content": "Moi rakas, mites koulussa meni?"
    \},\\
    \{
        "type": "TALK",
        "target": "Ellu",
        "content": "Ihan hyvin, mutta olisi aika nälkä."
    \},\\
    \{
        "type": "TALK",
        "target": "Äiti",
        "content": "Äiti on tässä tehnyt hyvää hernekeittoa."
    \},\\
    \{
        "type": "TALK",
        "target": "Ellu",
        "content": "Hyi! En halua hernekeittoa"
    \},\\
    \{
        "type": "TALK",
        "target": "Äiti",
        "content": "Pitää vähän maistaa"
    \},\\
    \{
        "type": "ACTION",
        "target": "Äiti",
        "content": "tyrkyttää Ellulle hernekeittoa"
    \},\\
    \{
        "type": "ACTION",
        "target": "Ellu",
        "content": "maistaa hernekeittoa ja kaatuu maahan"
    \},\\
    \{
        "type": "ACTION",
        "target": "Äiti",
        "content": "kauhistuu"
    \},\\
    \{
        "type": "TALK",
        "target": "Äiti",
        "content": "ELLU! Mitä sinulle kävi?!"
    \},\\
    \{
        "type": "TALK",
        "target": "Äiti",
        "content": "Täytyy soittaa hätäkeskukseen"
    \},\\
    \{
        "type": "ACTION",
        "target": "Äiti",
        "content": "soittaa hätäkeskukseen"
    \},\\
    \{
        "type": "TALK",
        "target": "Hätäkeskus",
        "content": "Hätäkeskus, mikä hätänä"
    \},\\
    \{
        "type": "TALK",
        "target": "Äiti",
        "content": "Ellu kaatui maahan"
    \},\\
    \{
        "type": "TALK",
        "target": "Äiti",
        "content": "Mikä osoite?"
    \},\\
    \{
        "type": "TALK",
        "target": "Äiti",
        "content": "Ellunkatu 13"
    \},\\
    \{
        "type": "TALK",
        "target": "Hätäkeskus",
        "content": "Laitetaaan apua tulemaan"
    \},\\
    \{
        "type": "ACTION",
        "target": "Ambulanssi kuski",
        "content": "saapuu paikalle"
    \},\\
    \{
        "type": "TALK",
        "target": "Ambulanssi kuski",
        "content": "Mikä hätänä?"
    \},\\
    \{
        "type": "ACTION",
        "target": "Äiti",
        "content": "(yrittää hysteeristesti sanoa jotain)"
    \},\\
    \{
        "type": "ACTION",
        "target": "Ambulanssi kuski",
        "content": "yrittää mitata äidin pulssia"
    \},\\
    \{
        "type": "TALK",
        "target": "Äiti",
        "content": "En minä vaan Ellu!"
    \},\\
    \{
        "type": "ACTION",
        "target": "Ambulanssi kuski",
        "content": "mittaa Ellun pulssia"
    \},\\
    \{
        "type": "TALK",
        "target": "Ambulanssi kuski",
        "content": "Se on kuollut, täytyy tilata hautaisurakoitsija paikalle"
    \},\\
    \{
        "type": "ACTION",
        "target": "Ambulanssi kuski",
        "content": "soittaa hautaisurakoitsijalle"
    \},\\
    \{
        "type": "ACTION",
        "target": "Hautaisurakoitsija",
        "content": "saapuu paikalle"
    \},\\
    \{
        "type": "TALK",
        "target": "Hautaisurakoitsija",
        "content": "Täytyy mitata, miten sopii arkkuun."
    \},\\
    \{
        "type": "ACTION",
        "target": "Hautaisurakoitsija",
        "content": "mittailee Ellua"
    \},\\
    \{
        "type": "TALK",
        "target": "Hautaisurakoitsija",
        "content": "Kyllä näyttäis olevan sen verran iso, että täytyy laittaa keskeltä poikki"
    \}\\
]

\chapter{Ohjeistus PaLM2-kielimallille}
\label{ch:palm2-guide}

Tulosteen tulee olla JSON-muodossa oleva lista, jossa jokaisella oliolla on
kolme attribuuttia: type, target ja content. Type on aina joko ACTION (kertoo,
että content on toimintaa) tai TALK (kertoo, että content on vuorosanoja).
Target on henkilö joka tekee tai puhuu asian. Content kertoo mitä tehdään tai
puhutaan.

\chapter{Ohjeistus Llama 3-kielimallille}
\label{ch:llama3-guide}

Kirjoita tarina vuorosanoilla. Tarinassa henkilöt saapuvat paikalle yksi tai
kaksi henkilö kerrallaan. Henkilöt ehtivät hetken jutella tai tehdä jotain
ennen kuin seuraava saapuu. Saapuvalla henkilöllä tulee olla syy, miksi saapuu
paikalle.

Tulosteen tulee olla JSON-muodossa oleva lista, jossa jokaisella oliolla on
kolme attribuuttia: type, target ja content. Type on aina joko ACTION (kertoo,
että content on toimintaa) tai TALK (kertoo, että content on vuorosanoja).
Target on henkilö joka tekee tai puhuu asian. Content kertoo mitä tehdään tai
puhutaan. Tulosteessa ei saa olla muua tekstiä kuin haluttu JSON.
