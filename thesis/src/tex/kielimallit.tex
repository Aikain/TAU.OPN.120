\chapter{Kielimallit}
\label{ch:kielimallit}

Nykyisin tekoälystä kuulee jatkuvasti puhuttavan jossain muodossa. Tekoälyn
merkityksestä tulevaisuudella kuulee jatkuvasti erilaisia arvioita ja
intensiivinen ja nopea kehitys alalla avaa jatkuvasti yhä uusia
mahdollisuuksia. Se, että mikä on tekoäly, ei kuitenkaan ole täsmällisesti
määritelty ja yksi mahdollinen määritys voisi olla Työ- ja elinkeinoministeriön
julkaisussa käytettävä määritelmä "Tekoäly tarkoittaa laitteita, ohjelmistoja
ja järjestelmiä, jotka kykenevät oppimaan ja tekemään päätöksiä lähes samalla
tavalla kuin ihmiset. Tekoälyn avulla koneet, laitteet, ohjelmat, järjestelmät
ja palvelut voivat toimia tehtävän ja tilanteen mukaisesti järkevällä tavalla"
\parencite{valtioneuvostoSuomenTekoalyaika}.

Yhtenä keinoa toteuttaa tekoälyä voidaan käyttää kielimalleja. Kielimallit on
yleensä koulutettu ymmärtämään ja tuottamaan luonnollista kieltä. Tänä päivänä
paljon esillä olevat kielimallit ovat niin sanottuja suuria kielimalleja,
joiden koulutus on pohjautunut merkittävän suureen aineistoon. Aineiston avulla
kielimallia voidaan opettaa ennustamaan, mitä sanoja tai lauseita aineistoin
mukaan annettaisiin vastauksena.

\section{Suuret kielimallit}

Suuret kielimallit tulivat tunnetuiksi suurelle osalle OpenAI:n marraskuussa
2022 julkaiseman ChatGPT-palvelun myötä \parencite{alma9911564814005973}.
Monille voikin tämän myötä jäädä kuva, että kielimalleilta voi kysyä jotain ja
ne tuottavat tekstiä vastauksena. Tätä monet niistä tekevätkin, mutta tänä
päivänä käyttötarkoituksia on keksitty useita. Perinteisen tekstin tuottamisen
lisäksi kielimalleja käytetään mm. kuvan, äänen ja videon tuottamiseen.
Kielimalleja on yhdistelty ja ne voivatkin ymmärtää useita eri syötteitä ja
tuottaa tuloksia sen mukaan.

Kielimalleilla on kuitenkin haasteita, joihin ne eivät kykene. Näitä ovat muun
muuassa kielen syvällisempien merkitysten ja ihmisten monimutkaisten
kommunikaatiomuotojen ymmärtäminen. Kielimallit eivät kykene abstraktiin
ajatteluun, tunteiden ymmärtämiseen tai syvälliseen kontekstuaaliseen
analyysiin. \parencite{haukkaJimiKuinkakielimallitOppivat}. Yksi sosiaalisessa
mediassa levinnyt esimerkki on hyvinkin yksinkertainen kysymys: "Kuinka monta
R-kirjainta on sanassa 'strawberry'?", johon moni kielimalli vastaa
virheellisesti kaksi \parencite{alberttechStrawberry}, jonka juurisyy pohjautuu
kielimallien tapaan käsitellä annettua syötettä.

Kielimallien antamassa tulosteessa voi olla virheitä useista syistä.
Koulutusaineistossa voi olla vinoumia, jonka myötä kielimalli voi tehdä
helposti vääriä johtopäätelmiä. Kielimallit antavat lisäksi välillä täysin
väärää tietoa, joka voi johtaa täysin harhaan ja käyttäjän onkin hankala
arvioida luotettavuutta ilman alkuperäistä lähdettä.
\parencite{akavaworksNakokulmiaTekoalyynOsa6}

\subsection{Tekstistä kuvaksi}

Ensimmäisiä selvästi suurempaan suosioon nousseita tekstistä kuvaksi -malleja,
eli malleja jotka tuottavat kuvan annetun tekstisyötteen perusteella, olivat
OpenAI:n DALL-E \parencite{openAIDallE} ja Stability AI:n Stable Diffusion
\parencite{stableDiffusionLaunch} sekä Midjourney
\parencite{twitter1547108864788553729}.

Yksi tuoreimmista on joulukuussa 2024 Googlen DeepMindin julkaisema uusin
versio Imagen 3 -mallista. Mallin kerrotaan kyekenevän luomaan korkealaatuisia
kuvia annetusta tekstisyötteestä. Kielimallien on arvioitu olevan paras kun
otetaan huomioon tasapaino kaikkien laatukriteetien välillä. Yksittäisissä
testeissä, kuvan visuaalisessa houkuttelevuudessa, MidJourney v6 on arvioitu
paremmaksi, mutta kokonaisuutta arvioiden Imagen 3 -mallin arvioidaan olevan
tällä hetkellä paras. \parencite{googleDeepmindImagen3_v3report}

Malleilla on kuitenkin yhä haasteita tiettyjen ominaisuuksien kanssa. Muun
muassa objektien tarkan lukumäärän tuottaminen on haastavaa sekä monet
kehotukset, jotka vaativat parempaa päättelyä ymmärtämisen suhteen, kuten
"talo on samankokoinen kuin kissa". \parencite{googleDeepmindImagen3_v3report}

\subsection{Tekstistä ääneksi}

Tekstistä ääneksi -mallit luovat annetun kuvauksen perusteella äänen.
\parencite{liu2023audioldmtexttoaudiogenerationlatent}. Näitä ei ole sekoittaa
tekstistä puheeksi -järjestelmiin jotka muuntavan tekstin puhutuksi.

Nämä mallit eivät välttämättä ole saaneet yhtä suurta näkyvyyttä kuin
visuaalisemmat mallit. Merkittävää edistystä on kuitenkin mallien kanssa
tapahtunut viimeisten vuosien aikana
\parencite{liu2023audioldmtexttoaudiogenerationlatent}.

Avoimeen lähdekoodiin pohjautuvista ratkaisuista esille tulee Suno:n luoma
Bark-malli. Malli tuottaa realistista, monikielistä puhetta sekä muita ääniä,
kuten musiikkia, taustamelua ja yksinkertaisia äänitehosteita. Malli kykenee
tuottamaan myös sanotonta viestintää, kuten naurua, huokailua ja itkua.
\parencite{githubSunoAiBark}

Osa malleista on erikoitustunut erityisesti musiikin tuottamiseen, jolloin
malleja saatetaankin kutsua tekstistä musiikiksi -malleiksi. Tällaisia malleja
ovat esimerkiksi Udio \parencite{udioAboutUs} ja MusicGen
\parencite{copet2024simplecontrollablemusicgeneration}.

\subsection{Tekstistä videoksi}

Helmikuussa 2024 OpenAi julkaisi Twitteriin useita tviittejä, jossa esiteltiin
lyhyitä videoita, jotka oli tuotettu antamalle Soralle, uudelle tekstistä
kuvaksi -mallille, lyhyehkö syöte siitä, mitä videossa halutaan tapahtuvan
\parencite{twitter1758192957386342435}. Ideana ei tuossa vaiheessa enää ollut
uutta, että toteutettaisiin videota tekstisyötteestä, mutta Soran tulokset
näyttivät, että se on todellakin mahdollista käytännössä eikä vain ideatasolla.

Soran kehityksen tautalla on ollut inspiraation ottaminen suurien kielimallien
toiminnasta. Suuret kielimallit on koulutettu internetin-laajuisella määrällä
tietoa ja LLM-paradigman menestys pohjautuukin tokeneihin. Vastaavasti Soralle
on kehitetty visuaalisia patcheja, joita on jo aiemmin nähty toimivan
visuaalisten toteutusten kanssa hyvin. Kuten suuret kielimallitkin, on Sora
tansformer-malli. Kuitenkin erona se, että Sora on nimenomaan diffuusio
transformer -malli. \parencite{openAISoraReport}

Soran kouluttamisessa on käytetty saatavilla olevaa dataa alkuperäisessä koossa
ilman leikkaamista vakiokokoisiksi. Tämä on yksi selvistä eroista verrattuna
muihin vastaaviin malleihin. Sen myötä on saatu useita etuja, kuten videoiten
tuottaminen eri kokoisina ja sisällön sijoittaminen paremmin näkyviin.
\parencite{openAISoraReport}

Kyseessä on tekstistä videoksi -malli, joten mallin on pelkän videon
tuottamisen lisäksi ymmärrettävä myös tekstiä. Tähän onkin Soran tapauksessa
käytetty OpenAI:n kehittämää tekniikkaa, jota DALL-E 3 käyttää. Myös GPT:n
avulla tekstisyötettä saadaan kuvailevammaksi ja mahdollistamaan sen myötä
parempien videoiden luontia. \parencite{openAISoraReport}

\section{Miten kielimalleja vertaillaan?}

Kielimallien määrän kasvaessa jatkuvasti on luonnollista, että kielimallit
kilpailevat siitä, mikä on paras kielimalli. Monien kielimallien julkaisuden
yhteydessä julkaistussa raportissa suurin osa sisällöstä onkin nimeen vertailua
ja statistiikkaa siitä miten kielimalli suoriutuu verrattuna muihin.
Raporteissa näkyvän vertailun voi jakaa kolmeen kategoriaan: kielimallien
vertailua aiempiin versioihin, kielimallien vertailua toisiin kielimalleihin
suorituskyvyn tai muun vastaavan mahdollisimman yksiselitteisen numerollisen
arvon avulla ja kielimallien vertailua erinäisten testien avulla, joilla
pyritään arvioimaan kielimallin kysyä mahdollisimman neutraalisti.

Kielimallien vertailuissa aiempiin malleihin Geminin kohdalla on toteutettu
useita eri suorituskykytestejä ja laskettu kuinka monessa testissä uudempi
versio on onnistunut paremmin ja sen myötä laskettu ns. voittosuhde. Näissä
suorituskykytestien määrä on vaihdellut ääniihin liittvien suorituskykytestien
viidestä aina tekstien 24 testiin ja keskeisten ominaisuuksien 50 testiin.
Määrät vaihtelevat myös riippuen minkä mallien välillä vertailua on tehty.
Vuoden 2024 helmikuun ja toukokuun mallien välisiä eroja on myös tarkemmin
eroteltu koostamalla kymmenkunta eri suorituskykytestiä ja testien antamia
pisteitä verrattu toisiinsa. \parencite{googleDeepmindGeminiv1_5report}

Geminin kohdalla on myös laskettu tehokkuutta mittaamalla sitä, minkä verran
kuluu keskimäärin aikaa yksittäiseen merkin tuottamiseen vastauksessa. Testissä
on annettu syötteitä englanniksi, japaniksi, kiinaksi ja ranskaksi. Mittaukset
on tehty käyttämällä Vertex AI:n ja OpenAI:n tarjoamia rajapintoja vertailteville
malleille. \parencite{googleDeepmindGeminiv1_5report} Tällä testillä on saatu
näkyviin niin sanottua raakaa dataa suorituskyvystä, mutta ottamatta kantaa
tuloksen varsiseen laatuun ja siten mallin suoriutumiseen. Vastaavia testejä
näkyykin huomattavasti vähemmän sillä ne eivät välttämättä yksinään kerro
mallien suoriutumisesta todelista kuvaa.

Yksi Gemini 1.5:n merkittävimmistä asioista vaikuttaisi olevan Googlen
esityksen \parencite{googleKeynote2024} perusteella mahdollisuus merkittävästi
suurempaan kontekstiin. Tätä korostuu myös mallissa tehdyssä raportissa
\parencite{googleDeepmindGeminiv1_5report}, jossa vertailu keskittyy paljon
toimivuuteen suuremmilla konteksteilla. Monessa tapauskessa muilla malleilla ei
ole edes mahdollisuutta kyseisen kokoiseen kontekstiin, joten vertailut
keskittyvätkin pitkälti näyttämään miten mallien suoriutuminen kestää
kontekstin kasvamisen myötä.

Kun Gemini 1:n raportissa \parencite{googleDeepmindGeminiv1report} vertailtiin
muihin malleihin, on vertailu Gemini 1.5:n
\parencite{googleDeepmindGeminiv1report} tapauksessa Geminin eri versioiden
välillä. Geminin ollessa kykenevä useisiin eri asioihin, on myös
suorituskykytestejä usealla eri osa-alueella, kuten tekstin kohdalla
kyvykkyys matematiikassa, tieteessä ja päättelyssä, monikielisyydessä,
sekä koodauksessa. Visuaalisuuteen liittyen puolestaan multimodaaliessa
päättelyssä, kaavioiden ja dokumenttien käsittelyssä, kuvien luonnollisuuteen
sek videoiden ymmärtämiseen. Äänten kanssa puolestaan puheen tunnistamiseen ja
kääntämiseen.

Clauden uusimpien mallien, Claude 3.5 Sonnet ja Claude 3.5 Haiku kanssa on
esille tuotu erityisesti tiettyjen suorituskykytestien tuloksia. Näissä
testeissä on vertailtu tuloksia Clauden versioiden sekä GPT:n ja Geminin
vastaavien mallien välillä. \parencite{anthropicClaudeSonnetAndHaiku35}
\parencite{anthropicClaudeSonnet} \parencite{anthropicClaudeHaiku} Tuloksien
ymmärtäminen näistä voi olla selkeämpää verrattuna mm. Geminin raportteihin
\parencite{googleDeepmindGeminiv1report}
\parencite{googleDeepmindGeminiv1_5report}, mutta on vertailu on jätetty vain
yksittäisten suorituskykytestien tulosten varaan.

Suorituskykytestejä on eri kielimallien raportonneissa
\parencite{anthropicClaudeSonnetAndHaiku35} \parencite{openAI2023GPT4}
\parencite{openAIGPT4o} \parencite{googleDeepmindGeminiv1_5report} käytetty
lukuisia, mutta eniten esille tulevat MMMU, GPQA, MATH, HumanEval, MGSM ja
DROP, jotka ovat esitelty jokaisen tutkitun kielimallien suorituskyvyn
raportoinnissa. Näiden lisäksi on useita eri suorityskykytestit, joista monet
tulevat esille useilla eri malleilla, vaikkeivat oletkaan esillä ihan
jokaisella mallilla. MMMU on suorituskykytesti, joka on suunniteltu
multimodaalisten mallien arviointiin massiivisiin, monialaisiin tehtäviin,
jotka vaativat korkeakoulutasoista tuntemusta ja päättelykykyä
\parencite{benchmarkMMMU}. GPQA on puolestaan kooste 448:sta haastavasta
monivalinta kysymyksestä biologian, fysiikan ja kemian alalla, johon tohtorin
tutkintoa tekevät saavat kaksi kolmasosaa oikein \parencite{benchmarkGPQA}.
Matemaattista ongelmanratkaisua mittaamaan on tehty MATH suorituskykyteksti,
jossa on 12 500 haastavaa kilpamatematiikan ongelmaa, jotka sisältävät
kattavan vaihettaisin ratkaisun, jotta kielimallien kykyä johdatteluun ja
selittävään vastaukseen voidaan myös parantaa \parencite{benchmarkMATH}.
HumanEval \parencite{benchmarkHumanEval} on arviointoisarja, joka sisältää
164 käsinkirjoitettuja ohjelmointiongelmaa. MSGM suorituskykytesti pohjautuu
250 peruskoulutason matematiikan ongelmaan \parencite{benchmarkMSGM}. DROP
puolestaan on 96 000 kysymystä sisältävä testi englanninkielen
luetunymmärtämiselle, jotka pyrkii tuomaan haastetta sijoittelemalla
tarvittava tieto useisiin kohteisiin, joista on kyettävä yhdistelemään
lopullinen vastaus.

Kielimallien suorituskykytestien lisäksi monesti kielimallien yhteydessä, tai
jopa niiden nimissä näkyy lukuja, jotka ovat usein miljardeissa. Nämä luvut
ovat useasti joko kielimallin koko tai käytetyn koulutusmateriaalin koko.
Kielimallien koosta puhuessa puhutaan parametreistä ja koulutusmateriaalin
kanssa tokeneista \parencite{kaplan2020scalinglawsneurallanguage}.

\section{Historia}

Tekoälyn historia alkaa 1950-luvun tienoilta, jolloin tänä päivänä tuntemamme
tietotekniikka oli vasta ihan alullaan. Ensimmäiset tekoälynratkaisut
suorittivat hyvin rajoitettuja tehtäviä, kuten ratkaisivat yksinkertaisia
matemaattisia laskutoimituksia. Vuonna 1956 järjestetyssä Dartmouth-
konferenssissa luotuun ehdotus, joka määäritteli perustavoitteet tekoälylle.
\parencite{alma9911564814005973}

1980-luvulla neuroverkot ja syväoppiminen avasivat uudenlaista potentiaalia
tekoälyille kun niiden nähtiin mahdollistavan muukin ratkaisu kuin vain
ohjelmistokehittäjän ennalta määrittämät tehtävät. Neuroverkoilla pyritään
matkikmaan ihmisten aivoja kun puolestaan syväoppimisella tuotiin useita
kerroksia, joissa jokainen kerros oppi datasta yhä monimutkaisempia piirteitä.
\parencite{alma9911564814005973}

Lähempänä nykypäivää käytettävissä olevan datan määrä on mahdollistanut sen
ettei tekoäly vain tekisi ihmisen määrittelemiä tehtäviä vaan kykenisi johonkin
oikeasti luovaan, kuten taiteeseen ja kirjoittamiseen.
\parencite{alma9911564814005973}
