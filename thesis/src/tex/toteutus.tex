\chapter{Kielimallit todellisessa käytössä}%
\label{ch:toteutus}

Tähän tulee tietoa toteutuksesta.

\section{Vertex AI: Gemini \& PaLM 2}

Vertex AI -alustan käyttöön on tarjolla kirjastoja useille eri
ohjelmointikielille \parencite{vertexAiGenerativeAiQuickstart}. Sovelluksen
backend on toteutettu javalla ja pakettimanagerina on maven, joten Vertex AI:n
javalle tekemän kirjasto saadaan käyttöön lisäämällä pom.xml-tiedostoon
esimerkin \ref{lst:vertexai-pom.xml} mukaisesti.

\begin{lstlisting}[
    basicstyle=\small,
    caption={google-cloud-vertexai riippuvuus pom.xml tiedostoon},
    label={lst:vertexai-pom.xml},
    language=xml,
]
<dependency>
    <groupId>com.google.cloud</groupId>
    <artifactId>google-cloud-vertexai</artifactId>
    <version>1.2.0</version>
</dependency>
\end{lstlisting}

Kevään 2024 aikana riippuvuuteen on tullut useita versioita, joista osa on
tuonut rikkovia muutoksia \parencite{mavenGoogleVertexAIAPI}. Kirjastoon
liittyvät esimerkit ovat tehty versiolle 1.2.0.

Sovellukseen on toteutettu backendin puolelle interface, jonka toteuttamalla
serviceä pystytään käyttämään helposti UI:n kanssa. Toteuttamalla interfacen
saadaan luotua tarina halutulla mallilla sekä jatkettua tarinaa. Esimerkissä
\ref{lst:gemini-service} on kyseisille metodeille tehdyt toteutukset, joissa
luodaan VertexAI:n kirjastosta löytyvät ChatSession ja laitetaan se talteen
niin tarinaa voidaan jatkaa myöhemmin.

\lstinputlisting[
    basicstyle=\small,
    caption={Backendiin toteutetun Gemini servicen tarinan luonti- ja jatko-metodien toteutukset},
    label={lst:gemini-service},
    language=java,
]{code/GeminiService.java}

PaLM 2:lle saadaan puolestaan toteutettua esimerkin \ref{lst:palm2-service}
mukaisesti. PaLM 2:lle ei ole tarjolla vastaavaa ChatSessionia, kuin Geminille,
joten on sille toteutettu vain tarinan luonti, muttei mahdollisuutta jatkaa
tarinaa.

\lstinputlisting[
    basicstyle=\small,
    caption={Backendiin toteutetun PaLM 2 servicen tarinan luontimetodin toteutukset},
    label={lst:palm2-service},
    language=java,
]{code/PaLM2Service.java}

\section{Claude 3.5 Sonnet}

Anthropic tarjoaa vain python kirjastoja Clauden käyttöön, joten sovellukseen
integrointi tehdään kutsuen suoraan saatavilla olevaa rajapintaa. Yksi
yksinkertainen vaihtoehto on käyttää openfeign:iä, joka saadaan käyttöön
lisäämällä `pom.xml`:n riippuvuuksiin \ref{lst:openfeign-pom.xml} mukainen
riippuvuus.

\begin{lstlisting}[
    basicstyle=\small,
    caption={spring-cloud-starter-openfeign riippuvuus pom.xml tiedostoon},
    label={lst:openfeign-pom.xml},
    language=xml,
]
<dependency>
    <groupId>org.springframework.cloud</groupId>
    <artifactId>spring-cloud-starter-openfeign</artifactId>
    <version>4.1.3</version>
</dependency>
\end{lstlisting}

Openfeign riippuvuuden lisäämisen jälkeen saadaan luotua interface
\ref{lst:claude-api-client}, johon toteutetaan tarpeelliset päätepisteet.

\lstinputlisting[
    basicstyle=\small,
    caption={Claude rajapinnan tarvittavien päätepisteiden toteuttaminen interfacena},
    label={lst:claude-api-client},
    language=java,
]{code/ClaudeApiClient.java}

Rajapinnan käyttäminen vaatii, että versio ja API avain on määritelty otsikko
-tietueisiin \parencite{anthropicAPIDocsVersions}
\parencite{anthropicAPIDocsGettingStarted}. Tämä saadaan toteutettua
mukautetulla RequestInterceptor toteutuksella esimerkin \ref{lst:claude-config}
mukaisesti, jossa API avain on luettu ympristömuuttujasta.

\lstinputlisting[
    basicstyle=\small,
    caption={Mukautettu RequestInterceptor toteutus},
    label={lst:claude-config},
    language=java,
]{code/ClaudeConfig.java}

Tämän jälkeen saadaan testattua esimerkin \ref{lst:claude-service} mukaisesti.

\lstinputlisting[
    basicstyle=\small,
    caption={Käynnistyksen yhteydessä lähetetään esimerkki viesti},
    label={lst:claude-service},
    language=java,
]{code/ClaudeService.java}
