\chapter{Kielimallien vertailu}%
\label{ch:vertailu}

\begin{itemize}
  \item Kuinka tärkeitä mitkäkin ominaisuudet, mitä painotetaan yms.
  \item Mitkä ovat kriteerit
  \begin{itemize}
    \item Onko jo käytettävissä (vapaasti käytettävissä vai vasta jotain kevyitä trialeita, joilla ei voi oikeasti tehdä mitään)?
    \item Onko käytettävissä millaisella työmäärällä?
    \item SaaS vs itse-hostattava tms.?
    \item Hinta
    \item Miten hyvin suoritutuu annetusta tehtävästä?
    \item Miten hyvin käsittelee "huonoja tilanteita" (esim. syöte on onnetonta, tuottaako mitään järkevää tai saako tiedon ettei syöte ollut hyvää tms.)
    \item Toimiiko suomeksi vai tarvitaanko erillinen kääntäminen?
  \end{itemize}
  \item Joku vertailutaulukko?
  \item  Käytännössä kaikki seuraavista on transformer arkkitehtuurin päälle rakennettuja etukäteen koulutettuja.
  \begin{itemize}
    \item GPT
    \begin{itemize}
      \item Generative pre-trained transformer
      \item OpenAI:n kehittämä
    \end{itemize}
    \item BERT
    \begin{itemize}
      \item Bidirectional Encoder Representations from Transformers
      \item Googlen kehittämä
      \item Encoder only (https://huggingface.co/learn/nlp-course/chapter1/5)
    \end{itemize}
    \item LLaMA
    \begin{itemize}
      \item Large Language Model Meta AI
      \item Metan kehittämä
      \item https://ai.meta.com/llama/
    \end{itemize}
    \item Bloom
    \begin{itemize}
      \item BigScience Large Open-science Open-access Multilingual Language Model
      \item Useamman tahon kehittämä
      \item Modified from Megatron-LM GPT2
      \item Decored only
    \end{itemize}
    \item Gemini
    \begin{itemize}
      \item LaMDA -> PaLM -> Gemini
      \item Googlen
      \item Decored only
      \item Vertautuu GPT-4 (V):hen
      \item https://ai.google.dev/
      \item Ei käytettävissä EU alueella
    \end{itemize}
    \item Chinchilla:
    \begin{itemize}
      \item Googlen
      \item Ei vielä julkisesti käytettävissä
    \end{itemize}
    \item FinBERT
    \item Finnish GPT-3
  \end{itemize}
  \item https://huggingface.co/AaltoSpeech
  \begin{itemize}
    \item CombFinnish-AED-CRDNN / Automatic Speech Recognition / Jonkin sortin puheentunnistusta?
  \end{itemize}
  \item https://huggingface.co/Helsinki-NLP
  \begin{itemize}
    \item simple-finnish-gpt3-xl / Text Generation / finetuned versio TurkuNLP/gpt3-finnish-xl:sta
  \end{itemize}
  \item https://huggingface.co/TurkuNLP
  \begin{itemize}
    \item bert-base-finnish-cased-v1, vertailujen mukaan parempi kuin Googlen multilingual (https://github.com/google-research/bert/blob/master/multilingual.md)
    \item löytyy myös uncased versio
    \item löytyy large variant
    \item toxicity versio, joka ilmeisesti tunnistaa 6 eri tyyppistä toxicityä
    \item QA versio
    \item bloom-finnish-176b / Text Generation / rakennettu bloom:in päälle sisältämään suomenkielisiä datasettejä
    \item gpt3-finnish-X / Text Generation / useampi erikokoinen GPT-3-malli, "puhtaita kielimalleja" eli ei ole hienosäädetty mm. vuoropuheluun tai kysymyksiin
    \item sbert-cased-finnish-paraphrase / Sentence Similarity / tunnistaa virkkeiden samankaltaisuutta
  \end{itemize}
  \item https://huggingface.co/Finnish-NLP
  \item https://huggingface.co/Kansallisarkisto
  \begin{itemize}
    \item finbert-ner / Token Classification / tunnistaa nimiä yms. tekstistä, hienosäädetty versio TurkuNLP/bert-base-finnish-cased-v1:sta
  \end{itemize}
  \item https://huggingface.co/tasks
  \item https://huggingface.co/models?pipeline\_tag=text-generation\&language=fi\&sort=trending
  \item https://huggingface.co/Finnish-NLP/llama-7b-finnish
  \item https://huggingface.co/TurkuNLP/gpt3-finnish-3B
\end{itemize}
