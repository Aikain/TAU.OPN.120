\chapter{Sovellus}%
\label{ch:sovellus}

Tähän tulee tietoa sovelluksesta.

\section{Käyttötarkoitus}

Mitä varten / Mihin sovellus tehdään?

Huomioitava Rippikoulusuunnitelma?

https://evl.fi/plus/seurakuntaelama/kasvatus/rippikoulu/suuri-ihme-rippikoulusuunnitelma/

Isostoiminnan linjaus?

https://evl.fi/plus/seurakuntaelama/kasvatus/rippikoulu/isostoiminta/

Jotain muuta?

\section{Toteutus}

Tähän yleinen kuvaus miten sovellus tehty.

\section{Sovelluksen esittelyä}

Perustoiminnallisuuden esittelyä yms.

* Seurakuntien ja riparien luonti

* Ohjaajinen, isosten, leiriläisten yms. lisääminen

* Ryhmien muodostus, satunnaisesti, mutta tietyillä säännöillä?

* Aikataulu (mahdollisuus reagoida muutoksiin, seurata tilannetta/kerätä statistiikkaa asioiden kestosta, antaa arvioita ajasta, iltaohjelman tilannetta voi seurata automaattisesti lauluja tunnistamalla?)

* Iltaohjelmien automaattinen generointi laulu-/leikkilistojen avulla

* Työkaluja kisojen pitämiseen (pistetaulut, ryhmien/parien arpominen, kiertojärjestys yms.)

* Muita työkaluja (isosvisa, teleprompteri)?

* Sketsit (apuna harjoitteluun: roolien jakaminen yms., sisällön generointi kielimalleilla yms. (esim. tarinan generointi haluttujen kuivisten ympärille))

* Ulkoläksykirjanpito

* Isosten/Ryhmien nakkilistat

* Raaamttu?
