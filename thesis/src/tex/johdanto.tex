\chapter{Johdanto}
\label{ch:johdanto}

Kielimallien käyttäminen on yleistynyt parin viime vuoden aikana huomattavasti
ja yhä useampi yritys ja organisaatio on ottanut kielimallit osaksi ellei jopa
kokonaan tulevaisuuden suunnitelmiksi. Isoilla teknologiajäteillä sekä suurelle
yleisölle ennen kielimallien suosion myötä saamaa julkisuutta hiukan
tuntemattomillakin teknologiafirmoilla on ollut jatkuva kilpailu siitä, kenen
kielimalli on kykenevin. Kielimalleja on vertailtu niin koulutettavan datan
määrällä kuin laitettu tekemään ihmisille suunnattuja tenttäjä tai muita kykyjä
mittaavia kokeita. Organisaatiorakenteitakin on pistetty uusiksi, jotta
kielimallien kehitys pystytään tekemään mahdollisimman tehokkaasti.

Tämän seurauksena tuleekin helposti kysymys: miten minä voisin käyttää
kielimalleja auttamaan tekemään niin sanotut raskaat työt? Voisiko kielimalli
luoda jotain luovaa nopeasti, jossa en ole itse onnistunut? Näiden kysymysten
pohtimiseen ja testaamiseen avautui sopiva tilaisuus kun eteen tuli tilanne,
jossa täytyi luoda nopeasti luovaa sisältöä ja muutaman henkilön parin tunnin
yrittämisen jälkeen ei sisältö ollut missään määrin kelvollista. Tässä työssä
onkin keskitytty luomaan työkalu, joka pyrkii estämään jatkossa kyseisen
tilanteen tapahtumisen. Työkalu tehdään osaksi diplomityön ohella tehtävää
sovellusta, joka pyrkii helpottamaan ja auttamaan seurakunnan rippileirin sekä
mahdollisesti muidenkin leirien vetäjien ja vapaaehtoisten koordinointia mm.
aikataulujen, vastuulistojen ja valmiiden materiaalien avulla.

Ensimmäisessä luvussa käydään lyhyesti läpi mitä kielimallit ovat, mitä niillä
voidaan nykyisin tehdä, miten niitä vertaillaan sekä lyhyesti katsotaan lyhyesti
kielimallien ja ylipäätään tekoälyn historiaa. Toisessa luvussa tuodaan esille
yleisiä tietoja muutamasta yleisestä kielimallilla, sekä lyhyt katsaus niiden
käytöstä ja hinnoittelusta, jotka vaikuttavat siihen, mitä kielimallia voidaan
lopullisessa toteutuksessa käyttää. Kolmannessa luvussa käydään läpi sovellusta,
siihen valittujen kielimallien integroinnin totutettamista sekä kevyt arviointi
kielimallien toimivuudesta.
