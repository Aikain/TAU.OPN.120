\chapter{Yhteenveto}
\label{ch:yhteenveto}

Tässä työssä kokeiltiin miten eri yleisiä kielimalleja saadaan integroitua
osaksi sovellusta ilman merkittävää työmäärää tai hienosäätöä. Työssä ei
pyritty käyttämään mitään edistynyttä arviointikriteeristöä toteutuksen
onnistumisessa tai toimivuudessa vaan hyvinkin yksinkertaisesti
ohjelmointitaitoisen harrastelijan näkökulmasta arviointia toimivuudesta.

Kielimalleja on tänä päivänä lukuisia ja uusia tulee jatkuvasti. Arviolta
puolet tässä työssä esitellyistä malleista on sellaisia, joita ei ollut vielä
olemassa työn aloitushetkellä. Kaiken kattavaa arviointia ja kokeilua onkin
mahdotonta tehdä, mutta työn pohjalta huomattiin, että usealla kielimallilla
on mahdollista luoda yksinkertaisesti toteutus, joka saatii tekemään sitä mitä
haluttiin.

Jotta kielimalleille saataisiin selviä eroja, olisi yksinkertaiset virheet
korjattava ja tämän jälkeen luotava kriteeristö, jolla arvioitaisiin
kielimallin tuottaman tulosteen laatua, sillä teknisestä näkökulmasta
kaikki kielimallit suoriutuivat hyvin. Tällöin todennäköisesti korostuisi se,
miten kielimallin oletusparametrit, kuten niin sanottu lämpötila, oikeasti
vaikuttaisivat, jolloin kielimallin käyttäminen siirtyisi kauemmaksi
yksinkertaisesta toteutuksesta.

Yleisten kielimallien käyttämisessä on kuitenkin yhä monia tunnettuja ongelmia.
Näistä työssä esille tuli kielimallin kyvyttömyys hallita määriä. Sovellukseen
yritettiin toteuttaa mahdollisuus määrittää tarinassa olevien henkilöiden määrä
antamalla kielimallille määrä numerona, mutta tällä ei ollut mitään merkitystä
kielimallin tulosteeseen.

Kielimalleja on kuitenkin hienosäädetty tekemään tiettyjä asioita, jonka myötä
paremmin erikoistunut kielimalli voisi olla parempi vaihtoehto kuin yleiset
kielimallit. Tällaisia voisivat olla esimerkiksi elokuvien käsikirjoittamiseen
käytettävät mallit.
