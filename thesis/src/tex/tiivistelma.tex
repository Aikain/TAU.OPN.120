Tässä työssä käydää läpi joitain tunnetuimmista kielimalleista. Osa
kielimalleista malleista valittiin integroitavaksi osaksi sovellusta, joka on
luotu helpottamaan isosten ja muuten seurakunnan henkilöstön koordinointia ja
työskentelyä rippileireillä.

Kielimallien osalta käytiin läpi hyvin perusasioita sekä työn kannalta
merkittävimmät asiat eli miten kielimalli saadaan integroitua osaksi toteutusta
sekä minkä verran kielimallin käyttäminen tulee maksamaan.

Kielimallien jatkuvan kehittymisen ja tilanteiden muuttumisen myötä lopulta
sovellukseen interoiduista kielimalleista lopulliseen vertailuun päätyi kaksi
kielimallia: Googlen Gemini 1.0 Pro ja Metan LLama3 405B. Kevyen vertailun
myötä LLama3 suoriutui huomattavasti luotettavammin ja varmemmin, mutta Gemini
puolestaan paikkasi puutteitaan paremmalla tehokkuudella.

Kielimallien suhteen on kuitenkin hankala verrata kielimalleja toisiinsa kun
vertailun tekohetkellä ilmestyy aina jo uusi ja parempi kielimalli. Työn aikana
osasta kielimalleista ilmestyi useita uusia versioita sekä integraatioon
käytettävistä kirjastoista uusia major-versioita. Työssä on kuitenkin käytetty
niitä versioita, jotka ovat olleet kyseisen kielimallin dokumentaation
suosittelemia sillä hetkellä, kun kielimalliin on perehdytty ensimmäisen
kerran.
