%%%%%%%%%%%%%%%%%%%%%%%%%%%%%%%%%%%%%%%%%%%%%%%%%%%%%%%%%%%
%% Congratulations, you've made an excellent choice
%% of writing your Tampere University thesis using
%% the LaTeX system. This document attempts to be
%% as complete a template as possible to let you focus
%% on the most important part: the writing itself.
%% Thus the details regarding the visual appearance
%% and even structure have already been worked out
%% for you!
%%
%% I sincerely hope you will find this template useful
%% in completing your thesis project. I've tried to
%% add comments (followed by the % sign) to clarify
%% the structure and purpose of some of the commands.
%% Most of the magic happens in the file tauthesis.cls,
%% which you are more than welcome to take a look at.
%% Just refrain from editing it in the most crucial
%% versions of the thesis!
%%
%% I wish you and your thesis project the best of luck!
%% If this template causes you trouble along the way
%% or if you've any suggestions for improving it,
%% please be in contact through GitHub
%% (<URL HERE>)
%%
%% Yours,
%%
%% Ville Koljonen
%%
%% PS. This template or its associated class file don't
%% come with a warranty. The content is provided as is,
%% without even the implied promise of fitness to the
%% mentioned purpose. You, as the author of the thesis,
%% are responsible for the entire work, including the
%% provided material. No one else is liable to you for
%% any damage inflicted on you or your thesis, were it
%% caused by using this template or not.
%%%%%%%%%%%%%%%%%%%%%%%%%%%%%%%%%%%%%%%%%%%%%%%%%%%%%%%%%%%

%%%%% NOTICE %%%%%
%% Please read through the entire template
%% (files under ./tex) to find all instructions.
%% It is possible that the attached pdf files
%% do not include the latest information.
%%%%%%%%%%%%%%%%%%

%%%%% INSTRUCTIONS FOR COMPILING THE DOCUMENT %%%%%
%% Overleaf: just click Recompile.
%% Terminal:
%%  1. pdflatex main.tex
%%  2. makeindex -s main.ist -t main.glg -o main.gls main.glo
%%  3. biber main
%%  4. pdflatex main.tex
%%  5. pdflatex main.tex
%% Similar sequence of commands is also required
%% in LaTeX specific editors.
%%%%%%%%%%%%%%%%%%%%%%%%%%%%%%%%%%%%%%%%%%%%%%%%%%%

%%% Set PDF version before doing anything else.

%%% set-pdf-version.tex
%
% This file is loaded by main.tex before any other operations, so that PDF
% version is set correctly for accessibility features.
%

\RequirePackage{ifluatex}

\def\mypdfminorversion{6}

\ifluatex

    \directlua {
        if pdf.getminorversion() \string~= \mypdfminorversion then
            if (status.pdf_gone and status.pdf_gone > 0)
            or (status.pdf_ptr and status.pdf_ptr > 0)
            then
                tex.error("PDF version cannot be changed anymore.")
            else
                pdf.setminorversion(\mypdfminorversion)
            end
        end
    }

\else

    \pdfminorversion=\mypdfminorversion

\fi


%%%%% METADATA %%%%%
%
% Always keep the following metadata up to date! This is important for your
% PDF file to comply to accessibility standards. (And yes, this information
% must remain here, before \documentclass[...]{...}.)

\def\myfititle{Kuvaava otsikko}
\def\myentitle{A Descriptive title}
\def\myauthor{Ville Nupponen}
\def\myfisubtitle{Tarkentava alaotsikko}
\def\myensubtitle{A Specifying Subtitle}
\def\myfithesistype{Diplomityö}
\def\myenthesistype{Master's thesis}
\def\myexaminers{Title1 Firstname1 Lastname1 \\ Title2 Firstname2 Lastname2 \\ ...}
\def\myfifacultyname{Informaatioteknologian ja viestinnän tiedekunta}
\def\myenfacultyname{Faculty of Information Technology and Communication Sciences}
\def\myfiprogrammename{Tietotekniikka}
\def\myenprogrammename{Computing Sciences}
\def\myfikeywords{avainsana1, avainsana2, ...}
\def\myenkeywords{keyword1, keyword2, ...}
\def\mylanguagecode{en-US}
\def\mysubject{A short description of the thesis subject.}
\def\myyear{2023}
\def\mymonth{06}
\def\myday{16}

%%%%% PREAMBLE %%%%%

%%%%% Document class declaration.
%
% The possible optional arguments are
%
%   finnish - thesis in Finnish (default)
%   english - thesis in English
%   numeric - citations in numeric style (default)
%   authoryear - citations in author-year style
%   apa - citations in APA 7 (available only in English)
%   ieee - citations in IEEE style (available only in English)
%   draft - for faster non-final works, also skips images
%           (recommended, remove in final version)
%   programs - if you wish to display code snippets
% Example: \documentclass[english, authoryear]{tauthesis}
%          thesis in English with author-year citations

\documentclass[finnish]{tauthesis}

%%% preamble.tex
%
% This file is for including LaTeX libraries or packages and defining your own
% commands.
%
% NOTE: The glossaries package loaded by tauthesis.cls throws a warning: No
% language module detected for 'finnish'. You can safely ignore this. All
% other warnings should be taken care of, before your thesis is submitted!

%%%%% Your packages.
%
% Before adding packages, see if they can be found in tauthesis.cls already.
% If you're not sure that you need a certain package, don't include it in the
% document! This can dramatically reduce compilation time.

% Graphs
% \usepackage{pgfplots}
% \pgfplotsset{compat=1.15}

% Subfigures and wrapping text
% \usepackage{subcaption}

%% Theorem environments and their numbering.
%
% Define both English and Finnish theorem types. These all follow the same
% counter. See the documentation of amsthm to see how these can be changed to
% suit your needs, if necessary.
%

\usepackage{amsthm}

\theoremstyle{definition}

%\newtheorem{definition}{Definition}[chapter]
%\newtheorem{theorem}[definition]{Theorem}

%\newtheorem{maaritelma}[definition]{Määritelmä}
%\newtheorem{lause}[definition]{Lause}

% Mathematics packages
\usepackage{mathtools, amssymb}
%\usepackage{bm}

% Chemistry packages
% \usepackage{chemfig}
% \usepackage[version=4]{mhchem}

% Text hyperlinking
% \usepackage{hyperref}
% \hypersetup{hidelinks}

% (SI) unit handling
% \usepackage{siunitx}

%\sisetup{
%    detect-all,
%    math-sf=\mathrm,
%    exponent-product=\cdot,
%    output-decimal-marker={,} % for theses in FINNISH!
%}

%%%%% Your commands.

% Print verbatim LaTeX commands
\newcommand{\verbcommand}[1]{\texttt{\textbackslash #1}}

% Command for formatting code.

\newcommand\code[1]{\texttt{#1}}

% Basic theorems in Finnish and in English.
% Remove [chapter] if you wish a simply
% running enumeration.
% \newtheorem{lause}{Lause}[chapter]
% \newtheorem{theorem}[lause]{Theorem}

% \newtheorem{apulause}[lause]{Apulause}
% \newtheorem{lemma}[lause]{Lemma}

% Use these versions for individually
% enumerated lemmas
% \newtheorem{apulause}{Apulause}[chapter]
% \newtheorem{lemma}{Lemma}[chapter]

% Definition style
% \theoremstyle{definition}
% \newtheorem{maaritelma}{Määritelmä}[chapter]
% \newtheorem{definition}[maaritelma]{Definition}
% examples in this style

%%%%% Glossary information.

% Use the following lines ONLY if you need more
% than one glossary. The first argument specifies
% a type label for the glossary and the second
% the displayed name.
% \newglossary*{symbs}{Symbols}
% \newglossary{label}{Displayed name}
% ...

\makeglossaries

% Use this line if using the default glossary.
% Otherwise comment out.

\loadglsentries[main]{tex/sanasto.tex}

% Use this line if using more than one glossary.
% Otherwise comment out.
% \loadglsentries[symbs]{tex/sanasto2.tex}

%%%%% Citation information.

% Commonly used bibliography modifications.
% Feel free to play around with them.

%\ExecuteBibliographyOptions{%
%sorting=none,
%maxbibnames=99,
%maxcitenames=2,
%giveninits=true,
%uniquename=init,
%sortcites,
%sortlocale=fin}

%\DefineBibliographyStrings{finnish}{%
%    in = {},
%    pages = {s.},
%    page = {s.}
%}
%\DefineBibliographyStrings{english}{%
%    in = {},
%    pages = {pp.},
%    page = {p.}
%}
%
%\DeclareNameAlias{sortname}{last-first}
%\DeclareNameAlias{author}{last-first}

%\DeclareFieldFormat[%
%    article,inbook,incollection,inproceedings,
%    patent,thesis,unpublished]{citetitle}{#1\isdot}
%\DeclareFieldFormat[%
%    article,inbook,incollection,inproceedings,
%    patent,thesis,unpublished]{title}{#1\isdot}
%\DeclareFieldFormat{pagetotal}{#1 \bibstring{page}}

%\AtBeginBibliography{\renewcommand*{\makelabel}[1]{#1\hss}}

%\DefineBibliographyExtras{english}{\let\finalandcomma=\empty}

\addbibresource{tex/references.bib}
 % You can add packages and define new commands in this file.

\begin{document}

%%%%% FRONT MATTER %%%%%

\frontmatter

%%%%% Thesis information and title page.

% Enable the use of @ character in command names.

\makeatletter

% The titles of the work. If there is no subtitle, leave the \myfisubtitle or
% \myensubtitle command arguments empty. Pass the title in the primary
% language as the first argument and its translation to the secondary language
% as the second.

\if@langenglish

    \title{\myentitle}{\myfititle}

\else

    \title{\myfititle}{\myentitle}

\fi

\if@langenglish

    \subtitle{\myensubtitle}{\myfisubtitle}

\else

    \subtitle{\myfisubtitle}{\myensubtitle}

\fi

% The author name.

\author{\myauthor}

% The examiner information. If your work has multiple examiners, replace with
%
%   \examiner[<label>]{<name> \\ <name>}
%
% where <label> is an appropriate (plural) label, e.g. Examiners or
% Tarkastajat, and <name>s are replaced by the examiner names, each on their
% separate line.

\examiner{\myexaminers}

% The finishing date of the thesis (YYYY-MM-DD).

\finishdate{\myyear}{\mymonth}{\myday}

% The type of the thesis (e.g. Kandidaatintyö or Master of Science Thesis) in
% the primary and the secondary languages of the thesis.

\if@langenglish

    \thesistype{\myenthesistype}{\myfithesistype}

\else

    \thesistype{\myfithesistype}{\myenthesistype}

\fi

% The faculty and degree programme names in the primary and the secondary
% languages of the thesis, respectively.

\if@langenglish

    \facultyname{\myenfacultyname}{\myfifacultyname}

\else

    \facultyname{\myfifacultyname}{\myenfacultyname}

\fi

\if@langenglish

    \programmename{\myenprogrammename}{\myfiprogrammename}

\else

    \programmename{\myfiprogrammename}{\myenprogrammename}

\fi

% The keywords of the thesis in the primary and the secondary languages of the
% thesis.

\if@langenglish

    \keywords{\myenkeywords}{\myfikeywords}

\else

    \keywords{\myfikeywords}{\myenkeywords}
\fi

% Make @ a regular letter again.

\makeatother

% Actually generate the title page based on the above commands.

\maketitle


%%%%% Abstracts and preface.
%
% Write the abstract(s) and the preface into a separate file for the sake of
% clarity. Pass the appropriate file name as the first argument to these
% commands. Put the \abstract in the primary language first and the
% \otherabstract in the secondary language second. Those who do not speak
% Finnish only need the first abstract. The second argument of the \preface
% command takes the place where the thesis was signed in.

\abstract{tex/tiivistelma.tex}

% \otherabstract{tex/abstract.tex}

\preface{tex/alkusanat.tex}{Tampereella}

%%%%% Table of contents.

\tableofcontents

%%%%% Lists of figures, tables, listings and terms.
%
% Print the lists of figures and/or tables. Uncomment either of these commands
% as required. Both are optional, but if there are many important
% figures/tables, listing them may be a good idea.

% \listoffigures
% \listoftables
% \lstlistoflistings

% Misc stuff related to how the glossary is displayed. You can especially
% tweak the lengths to suit you!

\glsaddall
\setglossarystyle{taulong}
\setlength{\glsnamewidth}{0.25\textwidth}
\setlength{\glsdescwidth}{0.75\textwidth}
\renewcommand*{\glsgroupskip}{}

% Print the default glossary of abbreviations, if necessary. Otherwise comment
% out. The appropriate Finnish variant is 'Lyhenteet'

\printglossary[title={Lyhenteet ja merkinnät}]

% Print more than one glossary with these lines. Otherwise comment out.

% \printglossary[type=symbs]
% \printglossary[type=label]
% ...

%%%%% MAIN MATTER %%%%%

\mainmatter

% Write each of the chapters of the thesis into a separate file for the sake
% of clarity. They can be \input as shown below. Give both the chapters and
% their files as descriptive names as possible.

\chapter{Johdanto}
\label{ch:johdanto}

Kielimallien käyttäminen on yleistynyt parin viime vuoden aikana huomattavasti
ja yhä useampi yritys ja organisaatio on ottanut kielimallit osaksi ellei jopa
kokonaan tulevaisuuden suunnitelmiksi. Isoilla teknologiajäteillä sekä suurelle
yleisölle ennen kielimallien suosion myötä saamaa julkisuutta hiukan
tuntemattomillakin teknologiafirmoilla on ollut jatkuva kilpailu siitä, kenen
kielimalli on kykenevin. Kielimalleja on vertailtu niin koulutettavan datan
määrällä kuin laitettu tekemään ihmisille suunnattuja tenttäjä tai muita kykyjä
mittaavia kokeita. Organisaatiorakenteitakin on pistetty uusiksi, jotta
kielimallien kehitys pystytään tekemään mahdollisimman tehokkaasti.

Tämän seurauksena tuleekin helposti kysymys: miten minä voisin käyttää
kielimalleja auttamaan tekemään niin sanotut raskaat työt? Voisiko kielimalli
luoda jotain luovaa nopeasti, jossa en ole itse onnistunut? Näiden kysymysten
pohtimiseen ja testaamiseen avautui sopiva tilaisuus kun eteen tuli tilanne,
jossa täytyi luoda nopeasti luovaa sisältöä ja muutaman henkilön parin tunnin
yrittämisen jälkeen ei sisältö ollut missään määrin kelvollista. Tässä työssä
onkin keskitytty luomaan työkalu, joka pyrkii estämään jatkossa kyseisen
tilanteen tapahtumisen. Työkalu tehdään osaksi diplomityön ohella tehtävää
sovellusta, joka pyrkii helpottamaan ja auttamaan seurakunnan rippileirin sekä
mahdollisesti muidenkin leirien vetäjien ja vapaaehtoisten koordinointia mm.
aikataulujen, vastuulistojen ja valmiiden materiaalien avulla.

Ensimmäisessä luvussa käydään lyhyesti läpi mitä kielimallit ovat, mitä niillä
voidaan nykyisin tehdä, miten niitä vertaillaan sekä lyhyesti katsotaan lyhyesti
kielimallien ja ylipäätään tekoälyn historiaa. Toisessa luvussa tuodaan esille
yleisiä tietoja muutamasta yleisestä kielimallilla, sekä lyhyt katsaus niiden
käytöstä ja hinnoittelusta, jotka vaikuttavat siihen, mitä kielimallia voidaan
lopullisessa toteutuksessa käyttää. Kolmannessa luvussa käydään läpi sovellusta,
siihen valittujen kielimallien integroinnin totutettamista sekä kevyt arviointi
kielimallien toimivuudesta.


% TODO: add chapters
%\input{tex/example.tex}

% Add chapters similarly.

\chapter{Yhteenveto}
\label{ch:yhteenveto}

Tässä työssä kokeiltiin miten eri yleisiä kielimalleja saadaan integroitua
osaksi sovellusta ilman merkittävää työmäärää tai hienosäätöä. Työssä ei
pyritty käyttämään mitään edistynyttä arviointi kriteeristöä toteutuksen
onnistumisessa tai toimivuudessa vaan hyvinkin yksinkertaisesti perus
ohjelmointitaidot omaavan harrastelijan näkökulmasta arviointia toimivuudesta.

Kielimalleja on tänä päivänä lukuisia ja jatkuvasti tulee uusia. Arviolta
puolet tässä työssä esitellyistä malleista on sellaisia, joita ei ollut vielä
olemassa työn aloitushetkellä. Kaiken kattavaa arviointia ja kokeilua onkin
mahdotonta tehdä, mutta työn pohjalta huomattiin, että usealla kielimallilla
on mahdollsita toteuttaa yksinkertaisesti totetus, joka saatii tekemään sitä
mitä haluttiin.

Jotta kielimalleille saataisiin selviä eroa, olisi yksinkertaiset virheet
korjattava ja tämän jälkeen luotava kriteeristö, jolla arvioitaisiin
kielimallin tuottaman tulosteen laatua, sillä teknisestä näkökulmasta
kaikki kielimallit suoriutuivat hyvin. Tällöin todennäköisesti korostuisi se,
miten kielimallin oletusparametrit, kuten niin sanottu lämpötila, oikeasti
vaikuttaisivat, jolloin kielimallin käyttäminen siirtyisi kaueammaksi
yksinkertaisesta toteutuksesta.

Yleisten kielimallien käyttämisessä on kuitenkin yhä monia tunnettuja ongelmia.
Näistä työssä esille tuli kielimallin kyvyttömyys osata hallita määriä.
Sovellukseen yritettiin toteuttaa mahdollisuus määrittää tarinassa olevien
henkilöiden määrä antamalla kielimallille määrä numerona, mutta tällä ei ollut
mitään merkitystä kielimallin tulosteeseen.

Kielimalleja on kuitenkin hienosäädetty tekemään tiettyjä asioita, jonka myötä
paremmin erikoistunut kielimalli voisi olla parempi vaihtoehto kuin yleiset
kielimallit. Tällaisia voisivat olla esimerkiksi elokuvien käsikirjoittamiseen
käytettävät mallit.


%%%%% Bibliography/references.

% Print the bibliography according to the information in ./tex/references.bib
% and the in-line citations used in the body of the thesis.

% \emergencystretch=2em
\printbibliography[heading=bibintoc]

%%%%% Appendices.

% Use only if it clarifies the structure of the document. Remember to
% introduce each appendix and its content.

\begin{appendices}

\chapter{Esimerkkiliite}%
\label{ch:liite}

Tämä on Esimerkkiliite.


\end{appendices}

\end{document}
